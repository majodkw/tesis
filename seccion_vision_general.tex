El sistema propuesto se concibe como un ecosistema modular de microservicios para el monitoreo, análisis y optimización del tráfico urbano en tiempo (casi) real, diseñado para operar en entornos de ciudad inteligente. La arquitectura se compone de cuatro módulos principales, cada uno con responsabilidades bien definidas y comunicándose mediante interfaces estandarizadas (APIs REST y mensajes JSON). Esta organización modular facilita la escalabilidad, el mantenimiento independiente de componentes y la integración con fuentes de datos heterogéneas.

\subsection{Módulos Principales}

El sistema se estructura en torno a cuatro módulos fundamentales:

\begin{itemize}
    \item \textbf{traffic-sim}: Módulo de simulación de tráfico urbano, que genera escenarios de prueba controlados y reproducibles para evaluar el comportamiento del sistema bajo diferentes condiciones de demanda vehicular. Permite emular redes viales con semáforos y flujos vehiculares variables.
    
    \item \textbf{traffic-control}: Módulo de orquestación que actúa como punto de entrada unificado al sistema. Recibe observaciones de tráfico, valida los datos, coordina las invocaciones a los servicios especializados de optimización y almacenamiento, y mantiene metadatos operativos en una base de datos relacional.
    
    \item \textbf{traffic-sync}: Módulo de análisis y optimización que procesa las métricas de tráfico recibidas, evalúa el nivel de congestión mediante técnicas de inteligencia computacional, y genera configuraciones optimizadas de tiempos semafóricos que buscan reducir demoras y mejorar el flujo vehicular.
    
    \item \textbf{traffic-storage}: Módulo de persistencia verificable que almacena datos del sistema en infraestructuras descentralizadas, garantizando trazabilidad, integridad e inmutabilidad de los registros históricos mediante el uso de tecnologías de almacenamiento distribuido y registro de auditoría.
\end{itemize}

\subsection{Flujo General de Información}

El flujo operativo del sistema sigue un ciclo cerrado de captura, análisis, optimización y registro:

\begin{enumerate}
    \item \textbf{Captura de datos}: El módulo de simulación (o sensores reales en un despliegue futuro) genera observaciones de tráfico que incluyen métricas agregadas por intersección semafórizada.
    
    \item \textbf{Validación y orquestación}: El módulo de control recibe las observaciones, valida su estructura y coherencia, y coordina el flujo de información hacia los servicios especializados.
    
    \item \textbf{Análisis y optimización}: El módulo de sincronización procesa las métricas, clasifica el estado de congestión y determina configuraciones de tiempos semafóricos que mejoren el desempeño operativo.
    
    \item \textbf{Persistencia verificable}: Los datos originales y los resultados de optimización se almacenan de forma distribuida, registrando identificadores de contenido en estructuras inmutables que permiten auditar y verificar el comportamiento histórico del sistema.
\end{enumerate}

\subsection{Ventajas de la Arquitectura Modular}

La separación de responsabilidades en módulos independientes proporciona varios beneficios arquitectónicos y operativos:

\begin{itemize}
    \item \textbf{Bajo acoplamiento}: Los módulos se comunican exclusivamente mediante interfaces REST bien definidas, lo que permite evolucionar o reemplazar componentes individuales sin afectar al resto del sistema.
    
    \item \textbf{Escalabilidad}: Cada módulo puede desplegarse de forma independiente en diferentes servidores o contenedores, permitiendo distribuir la carga computacional según las necesidades operativas.
    
    \item \textbf{Trazabilidad}: La combinación de almacenamiento distribuido con registros inmutables permite auditar el comportamiento del sistema, verificar la integridad de los datos y validar la conformidad con políticas operativas.
    
    \item \textbf{Reproducibilidad}: El uso de simulación permite evaluar el sistema bajo condiciones controladas, generando evidencias objetivas sobre el impacto de las estrategias de control propuestas.
    
    \item \textbf{Interoperabilidad}: La arquitectura basada en APIs REST facilita la integración con dispositivos IoT, plataformas de gestión urbana y otros sistemas externos que requieran consumir o producir datos de tráfico.
\end{itemize}

La Figura~\ref{fig:arquitectura_sistema_general} ilustra de forma esquemática la organización modular del sistema, mostrando los cuatro módulos principales y sus relaciones.

\begin{figure}[htbp]
    \centering
    \includegraphics[width=\textwidth]{images/arquitectura.pdf}
    \caption{Arquitectura general del sistema mostrando los cuatro módulos principales y el flujo de información.}
    \label{fig:arquitectura_sistema_general}
\end{figure}

En síntesis, esta arquitectura modular proporciona una plataforma flexible y escalable para la gestión inteligente de tráfico urbano, combinando capacidades de simulación, análisis adaptativo y persistencia verificable en un sistema distribuido que puede integrarse con infraestructuras reales de monitoreo vehicular. Los capítulos subsiguientes describen en detalle la implementación, las tecnologías subyacentes y el funcionamiento de cada módulo.
