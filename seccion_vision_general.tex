El sistema propuesto se concibe como un ecosistema modular de microservicios para el monitoreo, análisis y optimización del tráfico urbano en tiempo (casi) real, diseñado para operar sobre dispositivos de bajo costo y en entornos de ciudad inteligente. Cada módulo cumple una responsabilidad bien definida —simulación vehicular, análisis adaptativo de congestión, almacenamiento verificable y orquestación operativa— y se comunica con los demás mediante APIs REST y mensajes JSON, lo que facilita la integración con fuentes de datos heterogéneas y despliegues distribuidos. A diferencia de arquitecturas monolíticas o fuertemente centralizadas, esta organización permite escalar y evolucionar componentes de forma independiente, manteniendo al mismo tiempo una trazabilidad completa de los datos gracias al uso combinado de IPFS y contratos inteligentes sobre BlockDAG para el registro inmutable de resultados.

En términos funcionales, el flujo global de información sigue un ciclo cerrado. El módulo \emph{traffic-sim} utiliza SUMO y el protocolo TraCI para emular redes urbanas y generar observaciones macroscópicas de tráfico en intersecciones semaforizadas, incluyendo métricas como vehículos por minuto, velocidad media, tiempo medio de circulación, densidad y distribución por tipo de vehículo. Estas observaciones, o datos provenientes de sensores reales en un despliegue futuro, se envían al módulo \emph{traffic-control}, que actúa como puerta de entrada al sistema: valida la estructura y coherencia de los mensajes, registra metadatos en una base de datos relacional y coordina las llamadas hacia los servicios de análisis y almacenamiento. A continuación, \emph{traffic-control} delega el análisis al módulo \emph{traffic-sync}, que aplica un sistema de inferencia difusa tipo Mamdani para clasificar el nivel de congestión y un algoritmo PSO para proponer tiempos de verde y rojo óptimos, devolviendo una respuesta estructurada que resume las mejoras esperadas. Finalmente, tanto las observaciones originales como los resultados de optimización se envían al módulo \emph{traffic-storage}, que almacena los artefactos en IPFS y registra los identificadores de contenido en una red BlockDAG mediante contratos inteligentes, asegurando integridad, verificabilidad y disponibilidad distribuida de la información histórica.

Desde un punto de vista de arquitectura, esta división de responsabilidades proporciona varias ventajas. Por un lado, el acoplamiento entre captura de datos, análisis y almacenamiento es bajo: \emph{traffic-sim} y futuros dispositivos IoT sólo necesitan conocer el endpoint unificado de \emph{traffic-control}, mientras que los detalles de lógica difusa, PSO, IPFS y BlockDAG quedan encapsulados en módulos especializados. Por otro lado, la combinación de simulación reproducible con SUMO, optimización basada en técnicas de inteligencia computacional y persistencia verificable en infraestructuras descentralizadas permite evaluar el sistema bajo distintos escenarios de carga, al mismo tiempo que se generan evidencias auditables sobre su comportamiento, algo poco habitual en soluciones de gestión de tráfico urbano centradas únicamente en el control local o en plataformas cloud cerradas. La Figura~\ref{fig:arquitectura_sistema}, basada en el archivo \texttt{arquitectura.pdf}, resume gráficamente los módulos principales y las interacciones entre ellos, sirviendo como referencia visual del flujo de datos y de la integración con los servicios descentralizados de almacenamiento y verificación.

\begin{figure}[h]
    \centering
    \includegraphics[width=0.95\linewidth]{images/arquitectura.pdf}
    \caption{Arquitectura general del sistema, mostrando los módulos \emph{traffic-sim}, \emph{traffic-control}, \emph{traffic-sync} y \emph{traffic-storage}, así como su integración con IPFS y BlockDAG.}
    \label{fig:arquitectura_sistema}
\end{figure}
