% =============================================================================
% SECCIÓN 2.4: ALMACENAMIENTO DISTRIBUIDO: IPFS
% =============================================================================

\subsection{IPFS}
\label{sec:ipfs}

El InterPlanetary File System (IPFS) es un protocolo peer-to-peer de sistema de archivos distribuido que utiliza direccionamiento basado en contenido (content addressing) en lugar de ubicación (como HTTP) \cite{benet2014ipfs}. Esta arquitectura permite verificar la integridad de los datos mediante resúmenes criptográficos (hashes), aprovechar deduplicación automática y distribuir contenido entre múltiples nodos, lo que lo convierte en un complemento natural para blockchains y BlockDAG cuando se requiere almacenar archivos grandes fuera de la cadena manteniendo trazabilidad mediante identificadores de contenido.

% -----------------------------------------------------------------------------
\subsubsection{Arquitectura básica y modelo de archivos}
% -----------------------------------------------------------------------------

A alto nivel, IPFS puede verse como una combinación de varias ideas conocidas: tablas de hash distribuidas (DHT) para localizar contenido, intercambio de bloques entre pares, grafos acíclicos dirigidos de tipo Merkle (Merkle DAG) para estructurar los datos y un sistema de nombres mutable \cite{benet2014ipfs,ipfs_architecture}. Cada nodo IPFS participa en una red distribuida en la que puede almacenar, solicitar y reenviar bloques de datos identificados por su resumen criptográfico (hash).

Los archivos se fragmentan en bloques y se organizan en un Merkle DAG, donde cada nodo contiene datos o referencias a otros nodos, y el identificador del archivo completo se obtiene a partir del nodo raíz del DAG. Este identificador se conoce como Content Identifier (CID) y es un resumen criptográfico (hash) auto-descriptivo que codifica, además del valor del resumen criptográfico, el tipo de contenido y el algoritmo utilizado \cite{benet2014ipfs}. La dirección de un recurso en IPFS adopta típicamente la forma \texttt{/ipfs/\textless CID\textgreater/\textless path\textgreater}, donde el recorrido del DAG permite reconstruir el archivo original.

Desde el punto de vista de este trabajo, es suficiente entender que: (i) cualquier archivo que se agregue a IPFS se transforma en uno o más bloques; (ii) el sistema calcula un CID único en función del contenido; y (iii) mientras al menos un nodo mantenga esos bloques, cualquier otro nodo que conozca el CID puede recuperar el archivo. No es necesario profundizar en detalles de protocolos de transporte o variantes específicas de la DHT, dado que el uso previsto se limita a un nodo local o a servicios gestionados compatibles con IPFS.

% -----------------------------------------------------------------------------
\subsubsection{Identificadores de contenido (CIDs) y acceso a archivos}
% -----------------------------------------------------------------------------

Un indentificador de contenido (CID) es el identificador principal en IPFS y representa el resumen criptográfico (hash) del contenido de un objeto junto con metadatos mínimos sobre cómo interpretarlo \cite{benet2014ipfs}. El mismo contenido generará siempre el mismo CID, independientemente del nodo que lo procese, lo que permite verificar integridad de forma determinista: si el archivo se altera, el CID cambia. Esta propiedad es la base del direccionamiento por contenido y de la trazabilidad que se explota al combinar IPFS con tecnologías de registro inmutable.

En la práctica, el flujo básico de uso es el siguiente:

\begin{enumerate}
    \item \textbf{Subida}: el usuario agrega un archivo a un nodo IPFS (por ejemplo, un nodo local Kubo o un servicio compatible), y el nodo devuelve el CID asociado.
    \item \textbf{Registro}: la aplicación almacena ese CID como referencia persistente (por ejemplo, en una base de datos, en una blockchain o en un BlockDAG).
    \item \textbf{Descarga}: cuando se necesita recuperar el archivo, basta con proporcionar el CID a un nodo IPFS o a un gateway HTTP, que localizará los bloques correspondientes y reconstruirá el contenido.
\end{enumerate}

Este patrón es suficiente para las necesidades del presente trabajo, en las que IPFS se utiliza como almacén externo de archivos generados por el sistema (configuraciones, resultados, reportes), accesible mediante CIDs registrados en la capa de consenso.

% -----------------------------------------------------------------------------
\subsubsection{Fijación y puertas de enlace}
% -----------------------------------------------------------------------------

Por diseño, los nodos IPFS mantienen en cache el contenido que gestionan, pero pueden liberar bloques no referenciados cuando necesitan espacio. Para garantizar que determinados archivos permanezcan disponibles, se utiliza el mecanismo de \emph{pinning}, que marca objetos para que no sean eliminados automáticamente \cite{benet2014ipfs}. En un escenario simple con un único nodo local, basta con fijar (pin) los archivos relevantes para asegurar su persistencia mientras el nodo esté en funcionamiento.

Además de los nodos locales, existen puertas de enlace HTTP que permiten acceder a contenido IPFS desde un navegador o una aplicación web sin ejecutar un nodo completo. Un gateway recibe una URL de la forma \texttt{https://\textless gateway\_url\textgreater/ipfs/\textless CID\textgreater} y se encarga de resolver el CID en la red IPFS, recuperar los bloques y entregar el archivo vía HTTP. Para los objetivos de este trabajo, esto se traduce en la posibilidad de integrar el sistema con una interfaz web que sube archivos, obtiene CIDs y los recupera posteriormente a través de un nodo Kubo local o un servicio compatible (como plataformas de almacenamiento que exponen APIs tipo IPFS), sin necesidad de gestionar detalles internos de la red.

En resumen, IPFS se utiliza aquí como una capa de almacenamiento distribuido orientada a contenido, en la que la aplicación solo necesita conocer tres operaciones básicas: agregar archivos para obtener CIDs, fijar (pin) aquellos que deban persistir, y recuperar contenido a partir de su CID. Los aspectos de consenso, optimización de rutas o coordinación con otros nodos quedan fuera del alcance de este trabajo, ya que la integración se realiza a nivel de nodo local o servicio gestionado compatible con IPFS \cite{benet2014ipfs,ipfs_architecture}.
