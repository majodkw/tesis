% -------------------
% --- usepackages ---
% -------------------
\usepackage[utf8]{inputenc}
\usepackage{amsfonts}
\usepackage{csquotes} % For "csquotes missing" warning and babel-biblatex compatibility
\usepackage[
	spanish
]{babel} % Bibliography
\usepackage[
	backend=biber,
	style=numeric,
	sorting=none,
	maxbibnames=99
]{biblatex} % Bibliography
\usepackage{amsmath} % For equations.
\usepackage{amsthm} % For theorem, definition, lemma and proofs.
\usepackage{siunitx} % To write degree symbol for angles.
\usepackage{graphicx} % To add images
\usepackage{xcolor} % To add pdf_tex images
\usepackage{import} % To add pdf_tex images
\usepackage{changepage} % To adjustwidth
\usepackage{multirow} % To merge cells in table
\usepackage[
	Algoritmo
]{algorithm} % To write Algorithms
\usepackage[
	noend
]{algpseudocode} % To write Algorithms
\usepackage{caption} % To add subfigures
\usepackage[
	list=true
]{subcaption} % To add subfigures
\usepackage{booktabs} % To change weight of table lines
\usepackage{imakeidx} % To make index
\usepackage{geometry} % To set margin and page size
\usepackage{setspace} % To set interline spacing
\usepackage{chngcntr} % To enumerate captions by chapter
\usepackage{hyperref} % To autogenerate hyperlinks
\usepackage{amssymb} % For \boxplus.
\usepackage{environ} % To define appendix command
\usepackage{mathrsfs} % To cursive font
\usepackage{titlesec} % For custom chapter heading style
\usepackage{enumitem} % To define new type of list
\usepackage{multicol} % For multiple column in one page
\usepackage{cleveref} % For footnotes in multiple sentences
\usepackage{tikz} % For diagrams and figures
\usetikzlibrary{shapes,arrows,positioning,babel,calc}

% -------------------------
% --- makeindex options ---
% -------------------------
%     title:        header title
%     options= -s:  .ist file
%     intoc:        add to table of content
\makeindex[title=ÍNDICE ALFABÉTICO, options= -s index.ist, intoc]

% -------------------------
% --- bibliography file ---
% -------------------------
\addbibresource{bibliography.bib}

% ----------------------------
% --- document information ---
% ----------------------------
\title{Complejidad Computacional de Problemas de Empaquetamiento con Trominós}
\author{Javier Tadashi Akagi Fukushima}
\date{September 2019}

% -----------------------------
% --- page size and margins ---
% -----------------------------
\geometry{
	a4paper, 
	tmargin = 2.5cm,
	bmargin = 2cm,
	rmargin = 1.5cm,
	lmargin = 2.5cm
}

% ---------------------------------
% --- page format and numbering ---
% ---------------------------------
\pagestyle{plain}

% -----------------------
% --- interline space ---
% -----------------------
\onehalfspace
%\spacing{1.5}
%\doublespacing

% --------------------------------
% --- font style               ---
% ---  of definition, theorems ---
% --------------------------------
%     definition style: roman
\theoremstyle{definition}

% -------------------------
% --- caption numbering ---
% -------------------------
%     numbering by chapter:
%     <number of chapter>.<number of figure>
\counterwithin{figure}{chapter}
\counterwithin{equation}{chapter}
\counterwithin{table}{chapter}
\counterwithin{algorithm}{chapter}

% ----------------------------
% --- caption label format ---
% ----------------------------
%     subfigure caption format:
%     (<subfigure enumeration>) <description>
%\DeclareCaptionLabelFormat{subfigure}{Figura#1~\thefigure(#2):}
\DeclareCaptionLabelFormat{subfigureComment}{(#2):}
\DeclareCaptionLabelFormat{subfigureOnlyNumber}{(#2)}
%\DeclareCaptionLabelFormat{noCaption}{}
%\captionsetup[sub]{labelformat=subfigure}
\captionsetup[sub]{labelformat=subfigureOnlyNumber}

% ----------------------------------------
% --- table of content, list of figure ---
% ---  and list of table reedefinition ---
% ----------------------------------------
%     <title>                         Page
%     1. <chapter title> .... <page number>

\makeatletter
\renewcommand\tableofcontents{%
	\noindent
	\textbf{\large ÍNDICE} \hspace*{\fill} Página
	
	\@starttoc{toc}%
}
\makeatother

\makeatletter
\renewcommand\listoffigures{%
	\noindent
	\textbf{\large LISTA DE FIGURAS} \hspace*{\fill} Página

    \@starttoc{lof}%
}
\makeatother

\makeatletter
\renewcommand\listoftables{%
	\noindent
	\textbf{\large LISTA DE TABLAS} \hspace*{\fill} Página

    \@starttoc{lot}%
}
\makeatother

% ---------------------------
% --- new list definition ---
% ---------------------------
\newlist{listsymbol}{itemize}{1}
\setlist[listsymbol,1]{
	label=,
	labelwidth=1.9cm,
	align=parleft,
	itemsep=0.1\baselineskip,
	leftmargin=!
}

% ----------------------
% --- newenvironment ---
% ----------------------
\newenvironment{proofidea}{\begin{proof}[La idea de la demostración]}{\end{proof}}
\newenvironment{blockindent}{\vskip 0cm
\begin{adjustwidth}{.5cm}{.5cm}}{\end{adjustwidth}
\vskip .5cm}

% ------------------
% --- newcommand ---
% ------------------
\newcommand{\aztecdiam}{\mathcal{AD}} % To denote AD (aztec diamond)
\newcommand{\aztecrect}{\mathcal{AR}} % To denote AR (aztec rectangle)

\newcommand{\midbar}{\;\middle|\;} % To draw midline in set

\newcommand{\inout}[3]{ % To write a problem statement
	\vspace{0.2cm}
	\noindent
	\begin{tabular}{p{.13\textwidth} p{.006\textwidth} p{.775\textwidth}}
		\multicolumn{3}{l}{\scalebox{.85}{\textsc{#1}}}\\
		\hline
		\multicolumn{1}{|l}{ENTRADA} &:& \multicolumn{1}{p{.775\textwidth}|}{#2} \\
		\multicolumn{1}{|l}{SALIDA} &:& \multicolumn{1}{p{.775\textwidth}|}{#3} \\
		\hline
	\end{tabular}
	\vspace{0.2cm}
}
% \inout{<title>}{<input statement>}{<output statement>}
%
% will produce:
%
%  <title>
%  /--------------------------------\
%  | ENTRADA : <input statement>    |
%  | SALIDA  : <output statement>   |
%  \--------------------------------/

% -------------------------------------
% --- custom chapter heading format ---
% -------------------------------------
%     Label:  Chapter <chapter number>
%     Title:  <chapter title>
\newcommand{\chapterformatdefault}{
	\titleformat{\chapter}[display]
		{\LARGE } % Format for Label and Title
		{\vspace*{3.5cm}\MakeUppercase{\chaptertitlename} {\thechapter}} % Label
		{.2 cm} % Horizontal separation
		{\bfseries } % Code preceding to title body.
}

%     Title:  <chapter title>
\newcommand{\chapterformattitleonly}{
	\titleformat{\chapter}[display]
		{\LARGE } % Format for Label and Title
		{\vspace*{3.5cm}\MakeUppercase{\chaptertitlename} {\thechapter}} % Label
		{.2 cm} % Horizontal separation
		{\bfseries } % Code preceding to title body.
}

\newcommand{\trominodl}{\begin{gathered}\includegraphics[scale=.75]{images/L-tromino1.pdf}\end{gathered}} % To write a L-tromino in equation.
\newcommand{\trominodr}{\begin{gathered}\includegraphics[scale=.75]{images/L-tromino2.pdf}\end{gathered}} % To write a L-tromino in equation.
\newcommand{\trominour}{\begin{gathered}\includegraphics[scale=.75]{images/L-tromino3.pdf}\end{gathered}} % To write a L-tromino in equation.
\newcommand{\trominoul}{\begin{gathered}\includegraphics[scale=.75]{images/L-tromino4.pdf}\end{gathered}} % To write a L-tromino in equation.
\newcommand{\dominov}{\begin{gathered}\includegraphics[scale=.75]{images/dominoV.pdf}\end{gathered}} % To write a domino in equation.
\newcommand{\dominoh}{\begin{gathered}\includegraphics[scale=.75]{images/dominoH.pdf}\end{gathered}} % To write a domino in equation.
\newcommand{\floor}[1]{\left \lfloor #1 \right \rfloor} % To denote floor function

\newcommand\blfootnote[1]{ % To generate a footnote without marker
	\begingroup
	\renewcommand\thefootnote{}\footnote{#1}%
	\addtocounter{footnote}{-1}%
	\endgroup
}

% ---------------------------
% --- spanish translation ---
% ---------------------------

% -- Spanish theorems ---
\newtheorem{theorem}{Teorema}
\numberwithin{theorem}{chapter} % To number theorems according to the chapter
\newtheorem{definition}{Definición}
\numberwithin{definition}{chapter} % To number definitions according to the chapter
\newtheorem{lemma}{Lema}
\numberwithin{lemma}{chapter} % To number lemmas according to the chapter
\newtheorem{corollary}{Corolario}
\numberwithin{corollary}{chapter} % To number corollaries according to the chapter
\newtheorem{example}{Ejemplo}
\numberwithin{example}{chapter} % To number examples according to the chapter
\newtheorem{openproblem}{Problema abierto}
\numberwithin{openproblem}{chapter} % To number open problem according to the chapter
\addto\spanish{\renewcommand\proofname{Demostración}}

% -- Spanish algorithmic --
\algdef{S}[FOR]{ForEach}[1]{\algorithmicforeach\ #1\ \algorithmicdo}
\algnewcommand\algorithmicforeach{\textbf{para cada}}

\renewcommand{\algorithmicprocedure}{\textbf{procedimiento}}
\renewcommand{\algorithmicend}{\textbf{fin}}
\renewcommand{\algorithmicif}{\textbf{si}}
\renewcommand{\algorithmicthen}{\textbf{entonces}}
\renewcommand{\algorithmicelse}{\textbf{sino }}
\renewcommand{\algorithmicfor}{\textbf{desde}}
\renewcommand{\algorithmicforall}{\textbf{para todo}}
\renewcommand{\algorithmicdo}{\textbf{repetir}}
\renewcommand{\algorithmicwhile}{\textbf{mientras}}
\renewcommand{\algorithmicrepeat}{\textbf{repetir}}
\renewcommand{\algorithmicuntil}{\textbf{antes de}}
\renewcommand{\algorithmicreturn}{\textbf{retornar}}
%\renewcommand{\algorithmicrequire}{\textbf{Require:}}
%\renewcommand{\algorithmicensure}{\textbf{Ensure:}}
%\renewcommand{\algorithmicelsif}{\algorithmicelse\ \algorithmicif}
%\renewcommand{\algorithmicendif}{\algorithmicend\ \algorithmicif}
%\renewcommand{\algorithmicendfor}{\algorithmicend\ \algorithmicfor}
%\renewcommand{\algorithmicendwhile}{\algorithmicend\ \algorithmicwhile}
%\renewcommand{\algorithmicloop}{\textbf{loop}}
%\renewcommand{\algorithmicendloop}{\algorithmicend\ \algorithmicloop}

% -- Spanish list title --
\addto\captionsspanish{\renewcommand*\contentsname{ÍNDICE}}
\addto\captionsspanish{\renewcommand*\listfigurename{LISTA DE FIGURAS}}
\addto\captionsspanish{\renewcommand*\listtablename{LISTA DE TABLAS}}
