% =============================================================================
% SECCIÓN X: INGENIERÍA DE TRÁFICO – CONTROL SEMAFÓRICO
% =============================================================================

\section{Ingeniería de Tráfico y Control Semafórico}
\label{sec:traffic_engineering_signals}

La ingeniería de tráfico es la rama de la ingeniería del transporte que se ocupa del planeamiento, diseño, operación y gestión de instalaciones viales, con el objetivo de proporcionar un movimiento seguro y eficiente de personas y mercancías \cite{garber2019traffic,roess2019traffic}. Su enfoque combina el análisis de la interacción entre usuarios, vehículos e infraestructura con el estudio de parámetros macroscópicos de flujo, como volumen, velocidad y densidad, para evaluar y mejorar el desempeño de redes viales urbanas y periurbanas.

Dentro de este campo, el control semafórico constituye una de las herramientas centrales para regular el derecho de paso en intersecciones donde confluyen flujos vehiculares y peatonales conflictivos \cite{garber2019traffic,fhwa2008signal}. El diseño de planes de señalización semafórica abarca la definición de fases, la asignación de tiempos de verde, amarillo y rojo, y la coordinación entre intersecciones vecinas, con el fin de equilibrar seguridad, capacidad y niveles de servicio. La evolución desde esquemas de tiempos fijos hacia controles actuados y adaptativos refleja la necesidad de responder a patrones de demanda cada vez más variables, especialmente en entornos urbanos complejos.

\subsection{Semáforos para el Control del Tránsito de Vehículos}
\label{subsec:semaforos_control_vehiculos}

Los semáforos constituyen el principal dispositivo de control del derecho de paso en intersecciones urbanas y periurbanas con conflictos significativos de circulación vehicular y peatonal \cite{mopc2019manual}. En el Manual de Carreteras del Paraguay se establece que los semáforos para el control del tránsito de vehículos deben emplearse para asignar tiempos de paso y de detención a los distintos movimientos que confluyen en la intersección, con el objetivo de reducir conflictos, ordenar las maniobras y aumentar la seguridad vial \cite{mopc2019manual}. Esta función se complementa con el control de flujos peatonales y, cuando corresponde, con fases específicas para giros, carriles exclusivos y prioridades especiales (por ejemplo, transporte público o vehículos de emergencia) \cite{mopc2019manual}.

Desde el punto de vista de la ingeniería de tráfico, un semáforo vehicular puede interpretarse como un sistema de control discreto en el tiempo que alterna estados de indicación verde, amarilla y roja para cada grupo de movimientos, de forma tal que en ningún instante se autoricen simultáneamente movimientos incompatibles \cite{mopc2019manual,garber2019traffic}. El diseño del plan de fases considera la geometría de la intersección, los volúmenes de tránsito por aproximación, la presencia de giros a la izquierda y a la derecha, los flujos peatonales y, en su caso, la coordinación con semáforos próximos en el corredor \cite{mopc2019manual,roess2019traffic}. Las indicaciones luminosas verde, amarilla y roja poseen significado normado asociado a las obligaciones legales de detenerse o avanzar, y se complementan con flechas direccionales cuando es necesario discriminar maniobras permitidas dentro de una misma aproximación \cite{mopc2019manual}.

En la normativa paraguaya se especifican además criterios de ubicación, número de caras y área de influencia de cada cabezal semafórico, de modo que las indicaciones resulten claramente visibles para todos los usuarios a los que se dirigen \cite{mopc2019manual}. Estos lineamientos aseguran que el dispositivo de control implementado por el sistema desarrollado en el presente trabajo sea compatible con la infraestructura semafórica existente y respetuoso de las disposiciones vigentes en el país.

% -----------------------------------------------------------------------------


\subsection{Semáforos de Tiempos Fijos o Predeterminados}
\label{subsec:semaforos_tiempo_fijo}

Los semáforos de tiempos fijos o predeterminados operan con un plan de señales cuya secuencia de fases y tiempos se define previamente y se repite cíclicamente, sin interacción directa con el tránsito en tiempo real \cite{mopc2019manual}. El Manual de Carreteras del Paraguay indica que estos equipos asignan a cada fase un tiempo de verde y un tiempo intermedio (amarillo y, si corresponde, tiempo de despeje en rojo) calculados a partir de volúmenes de tránsito de diseño, análisis de capacidad y niveles de servicio objetivo \cite{mopc2019manual}. Los tiempos de fase se mantienen constantes durante el período de operación del plan, aunque se contempla el uso de distintos planes fijos según la franja horaria (pico de la mañana, valle, pico de la tarde, periodo nocturno) \cite{mopc2019manual,garber2019traffic}.

La principal ventaja de los semáforos de tiempos fijos reside en su simplicidad de diseño, facilidad de implementación y previsibilidad del patrón de señales, lo que se traduce en menores requerimientos de equipamiento (no se requieren detectores de vehículos) y en un mantenimiento relativamente sencillo \cite{garber2019traffic,roess2019traffic}. Sin embargo, este tipo de control presenta limitaciones en entornos con variaciones significativas de la demanda, como redes urbanas con fuertes diferencias entre horas pico y valle o con eventos recurrentes que alteran los patrones de flujo, ya que el plan fijo no se adapta a la ocupación real de las aproximaciones y puede generar tiempos de verde desaprovechados o demoras excesivas en ciertas corrientes \cite{papageorgiou2003review,stevanovic2010adaptive}.

Diversos estudios en ingeniería de tráfico han mostrado que, en condiciones de demanda altamente variable, los esquemas de control fijo tienden a producir mayores tiempos de retraso medio, un mayor número de paradas y colas más extensas que los sistemas actuados o adaptativos \cite{papageorgiou2003review,stevanovic2010adaptive}. Este comportamiento motiva la búsqueda de estrategias de control más flexibles, como las basadas en lógica difusa o algoritmos de optimización metaheurística, que constituyen el foco del trabajo.


% -----------------------------------------------------------------------------


\subsection{Semáforos Accionados por el Tránsito}
\label{subsec:semaforos_accionados_transito}

Los semáforos accionados por el tránsito utilizan información en tiempo real captada por detectores (por ejemplo, lazos inductivos, sensores de radar o cámaras de video) ubicados en las aproximaciones para modificar la duración de las fases dentro de límites mínimos y máximos definidos \cite{mopc2019manual}. El Manual de Carreteras del Paraguay distingue entre control parcialmente actuado (al menos una aproximación bajo control por detectores) y totalmente actuado (todas las aproximaciones controladas por demanda), estableciendo criterios de diseño para la ubicación de detectores, la distancia respecto de la línea de detención y los períodos iniciales mínimos y extensiones de verde en función de la velocidad que comprende el 85\,\% del tránsito en el acceso \cite{mopc2019manual}.

En estos sistemas, el controlador decide en cada ciclo si extiende o termina el verde de una fase atendiendo a la detección de vehículos, lo que permite reducir tiempos de verde desaprovechados en movimientos con baja demanda y asignar capacidad adicional a las aproximaciones más cargadas \cite{roess2019traffic,papageorgiou2003review}. La literatura sobre control semafórico actuado y adaptativo destaca que este tipo de sistemas mejora el desempeño frente a planes fijos, especialmente en condiciones de demanda fluctuante, al disminuir el retraso medio por vehículo y el número de paradas, y al reducir la probabilidad de colas críticas en aproximaciones dominantes \cite{papageorgiou2003review,stevanovic2010adaptive}. 

El uso de detectores también permite implementar estrategias avanzadas como extensiones de verde para “limpiar” colas residuales, prioridad al transporte público o prioridad a vehículos de emergencia, resultando particularmente relevante en corredores urbanos con alta variabilidad de flujos \cite{papageorgiou2003review,gartner2001arterial}. Estos conceptos constituyen la base sobre la cual se articula el sistema de control inteligente propuesto en este trabajois, donde un controlador difuso tipo Mamdani y algoritmos de optimización como Particle Swarm Optimization (PSO) se integran para ajustar dinámicamente parámetros de operación semafórica en respuesta a condiciones de tráfico cambiantes.


% -----------------------------------------------------------------------------


\subsection{Parámetros Clásicos de Diseño de Tiempos Semafóricos}
\label{subsec:parametros_diseno_tiempos}

En el diseño clásico de control semafórico, la configuración de una intersección se describe mediante un conjunto reducido de parámetros numéricos que determinan su desempeño operativo: la longitud de ciclo, la repartición de verdes por fase (splits), los tiempos de amarillo y todo–rojo, y los tiempos perdidos en cada etapa del ciclo \cite{garber2019traffic,roess2019traffic}. La longitud de ciclo $C$ representa la duración total de una repetición completa de las fases del semáforo, mientras que los splits determinan qué fracción de ese ciclo se asigna al verde efectivo de cada movimiento o grupo de movimientos. Los tiempos de amarillo y todo–rojo se definen para advertir el fin del derecho de paso y garantizar el despeje seguro de la intersección antes de otorgar el verde a flujos conflictivos, y los tiempos perdidos agrupan intervalos en los que, por razones de seguridad o arranque, la capacidad efectiva de la intersección es menor a la teórica \cite{garber2019traffic}.

Estos parámetros influyen directamente en la capacidad y el retraso en cada aproximación. Para una misma demanda, ciclos más largos tienden a aumentar la capacidad por fase pero también pueden incrementar el tiempo medio de espera si los usuarios deben esperar largos intervalos de rojo, mientras que ciclos muy cortos reducen el retraso pero pueden volverse ineficientes si el tiempo perdido ocupa una proporción significativa del ciclo \cite{roess2019traffic,papageorgiou2003review}. La capacidad se relaciona con la proporción de verde asignada a cada movimiento y con la saturación de los carriles, en tanto que el retraso medio por vehículo, el número de paradas y la longitud de colas son métricas fundamentales para evaluar la calidad de servicio de un plan de tiempos \cite{garber2019traffic,fhwa2008signal}.

Métodos de diseño tradicional, como la formulación de Webster, proporcionan expresiones para estimar un ciclo ``óptimo'' fijo a partir de los volúmenes de diseño y de los tiempos perdidos, buscando un compromiso entre capacidad y retraso medio \cite{webster1958traffic,fhwa2008signal}. En la práctica, estos valores se calculan para condiciones de flujo representativas y se implementan como planes de tiempo fijo, eventualmente diferenciados por franja horaria. En contraste, el enfoque desarrollado en este trabajo utiliza un sistema de inferencia difusa tipo Mamdani y un algoritmo de optimización PSO para ajustar dinámicamente la longitud de ciclo efectiva, los splits y otros parámetros de operación en función de mediciones de tráfico en tiempo real, con el objetivo de reducir retrasos y mejorar el desempeño frente a condiciones de demanda variables \cite{papageorgiou2003review,stevanovic2010adaptive}.


% -----------------------------------------------------------------------------


\subsection{Coordinación de Semáforos y Progresión de Pelotones}
\label{subsec:coordinacion_semaforos}

En redes urbanas, los semáforos rara vez operan de manera completamente aislada; por el contrario, se busca coordinar varios equipos a lo largo de un corredor para facilitar la progresión de pelotones de vehículos y reducir el número de paradas sucesivas \cite{garber2019traffic,roess2019traffic}. La coordinación se logra ajustando parámetros como la longitud de ciclo común entre intersecciones, los offsets (desplazamientos temporales entre inicios de verde) y, en algunos casos, la secuencia de fases, de modo que un grupo de vehículos que parte con verde en una intersección encuentre verdes sucesivos en las siguientes, configurando la llamada ``onda verde'' \cite{fhwa2008signal,erdogan2013coordination}. Estos esquemas buscan compatibilizar la operación de múltiples nodos de control, sacrificando a veces el óptimo local de una intersección en beneficio del desempeño global del corredor.

La coordinación introduce dependencias fuertes entre los parámetros de distintas intersecciones: cambios en la longitud de ciclo o en los splits de una intersección pueden degradar la progresión en tramos adyacentes, generando nuevas demoras o incrementando el número de paradas en el corredor \cite{erdogan2013coordination,roess2019traffic}. Por ello, el diseño de planes coordinados suele apoyarse en herramientas de simulación y optimización que consideran simultáneamente varias intersecciones, y en prácticas de monitoreo operativo que permiten ajustar offsets y ciclos en respuesta a cambios de demanda. En este contexto, la idea de un sistema de control y monitoreo distribuido, apoyado en tecnologías como BlockDAG e IPFS para registrar configuraciones y métricas de desempeño, resulta especialmente pertinente: aunque el presente trabajo se centre en la optimización a nivel de una intersección, la misma arquitectura puede escalar a corredores coordinados donde las decisiones de tiempo de verde, ciclo y offset deben gestionarse de forma conjunta y trazable.


% -----------------------------------------------------------------------------


\subsection{Consideraciones Operativas y Métricas de Desempeño}
\label{subsec:metricas_desempeno}

La evaluación de un plan de tiempos semafóricos se basa en un conjunto de métricas operativas que permiten cuantificar su impacto sobre los usuarios de la vía. Entre las más utilizadas se encuentran el retraso medio por vehículo, el número de paradas, la longitud de colas y el nivel de servicio (Level of Service, LOS); en estudios recientes se añaden además indicadores de consumo de combustible y emisiones contaminantes \cite{garber2019traffic,roess2019traffic,papageorgiou2003review}. El retraso medio mide la diferencia entre el tiempo de viaje real y el tiempo teórico sin interferencias, mientras que el número de paradas y la longitud de colas caracterizan la discontinuidad del flujo y la probabilidad de bloqueo de accesos o de intersecciones adyacentes \cite{garber2019traffic,fhwa2008signal}.

El nivel de servicio resume de forma cualitativa la calidad de operación en categorías que van desde ``A'' (pocas demoras, operación confortable) hasta ``F'' (congestión severa), definidas a partir de rangos de retraso medio por vehículo y otras variables \cite{roess2019traffic,fhwa2008signal}. Estas métricas permiten comparar diferentes esquemas de control (planes fijos, control actuado, control adaptativo) bajo condiciones de demanda similares y constituyen la base para justificar cambios en la programación de semáforos desde una perspectiva de ingeniería. En el contexto de tu implementación, ese párrafo se puede ajustar así:

En el contexto de este trabajo, el desempeño de los planes semafóricos se evalúa a partir de indicadores derivados de la salida del sistema difuso de congestión (valor numérico de congestión y su categoría lingüística), junto con variables macroscópicas como volumen por minuto, velocidad media y densidad vehicular. El controlador difuso Mamdani proporciona una medida agregada de congestión a partir de estas variables, y el algoritmo PSO ajusta los tiempos de verde para minimizar directamente dicho índice de congestión y mejorar las condiciones de flujo respecto a una asignación base de tiempos calculada mediante fórmulas clásicas de ingeniería de tráfico.
