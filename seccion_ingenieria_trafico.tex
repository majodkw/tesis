% =============================================================================
% SECCIÓN X: INGENIERÍA DE TRÁFICO – CONTROL SEMAFÓRICO
% =============================================================================

\section{Ingeniería de Tráfico y Control Semafórico}
\label{sec:traffic_engineering_signals}

La ingeniería de tráfico es la rama de la ingeniería del transporte que se ocupa del planeamiento, diseño, operación y gestión de instalaciones viales, con el objetivo de proporcionar un movimiento seguro y eficiente de personas y mercancías \cite{garber2019traffic,roess2019traffic}. Su enfoque combina el análisis de la interacción entre usuarios, vehículos e infraestructura con el estudio de parámetros macroscópicos de flujo, como volumen, velocidad y densidad, para evaluar y mejorar el desempeño de redes viales urbanas y periurbanas.

Dentro de este campo, el control semafórico constituye una de las herramientas centrales para regular el derecho de paso en intersecciones donde confluyen flujos vehiculares y peatonales conflictivos \cite{garber2019traffic,fhwa2008signal}. El diseño de planes de señalización semafórica abarca la definición de fases, la asignación de tiempos de verde, amarillo y rojo, y la coordinación entre intersecciones vecinas, con el fin de equilibrar seguridad, capacidad y niveles de servicio. La evolución desde esquemas de tiempos fijos hacia controles actuados y adaptativos refleja la necesidad de responder a patrones de demanda cada vez más variables, especialmente en entornos urbanos complejos.

\subsection{Semáforos para el Control del Tránsito de Vehículos}
\label{subsec:semaforos_control_vehiculos}

Los semáforos constituyen el principal dispositivo de control del derecho de paso en intersecciones urbanas y periurbanas con conflictos significativos de circulación vehicular y peatonal \cite{mopc2019manual}. En el Manual de Carreteras del Paraguay se establece que los semáforos para el control del tránsito de vehículos deben emplearse para asignar tiempos de paso y de detención a los distintos movimientos que confluyen en la intersección, con el objetivo de reducir conflictos, ordenar las maniobras y aumentar la seguridad vial \cite{mopc2019manual}. Esta función se complementa con el control de flujos peatonales y, cuando corresponde, con fases específicas para giros, carriles exclusivos y prioridades especiales (por ejemplo, transporte público o vehículos de emergencia) \cite{mopc2019manual}.

Desde el punto de vista de la ingeniería de tráfico, un semáforo vehicular puede interpretarse como un sistema de control discreto en el tiempo que alterna estados de indicación verde, amarilla y roja para cada grupo de movimientos, de forma tal que en ningún instante se autoricen simultáneamente movimientos incompatibles \cite{mopc2019manual,garber2019traffic}. El diseño del plan de fases considera la geometría de la intersección, los volúmenes de tránsito por aproximación, la presencia de giros a la izquierda y a la derecha, los flujos peatonales y, en su caso, la coordinación con semáforos próximos en el corredor \cite{mopc2019manual,roess2019traffic}. Las indicaciones luminosas verde, amarilla y roja poseen significado normado asociado a las obligaciones legales de detenerse o avanzar, y se complementan con flechas direccionales cuando es necesario discriminar maniobras permitidas dentro de una misma aproximación \cite{mopc2019manual}.

En la normativa paraguaya se especifican además criterios de ubicación, número de caras y área de influencia de cada cabezal semafórico, de modo que las indicaciones resulten claramente visibles para todos los usuarios a los que se dirigen \cite{mopc2019manual}. Estos lineamientos aseguran que el dispositivo de control implementado por el sistema desarrollado en el presente trabajo sea compatible con la infraestructura semafórica existente y respetuoso de las disposiciones vigentes en el país.

% -----------------------------------------------------------------------------


\subsection{Semáforos de Tiempos Fijos o Predeterminados}
\label{subsec:semaforos_tiempo_fijo}

Los semáforos de tiempos fijos o predeterminados operan con un plan de señales cuya secuencia de fases y tiempos se define previamente y se repite cíclicamente, sin interacción directa con el tránsito en tiempo real \cite{mopc2019manual}. El Manual de Carreteras del Paraguay indica que estos equipos asignan a cada fase un tiempo de verde y un tiempo intermedio (amarillo y, si corresponde, tiempo de despeje en rojo) calculados a partir de volúmenes de tránsito de diseño, análisis de capacidad y niveles de servicio objetivo \cite{mopc2019manual}. Los tiempos de fase se mantienen constantes durante el período de operación del plan, aunque se contempla el uso de distintos planes fijos según la franja horaria (pico de la mañana, valle, pico de la tarde, periodo nocturno) \cite{mopc2019manual,garber2019traffic}.

La principal ventaja de los semáforos de tiempos fijos reside en su simplicidad de diseño, facilidad de implementación y previsibilidad del patrón de señales, lo que se traduce en menores requerimientos de equipamiento (no se requieren detectores de vehículos) y en un mantenimiento relativamente sencillo \cite{garber2019traffic,roess2019traffic}. Sin embargo, este tipo de control presenta limitaciones en entornos con variaciones significativas de la demanda, como redes urbanas con fuertes diferencias entre horas pico y valle o con eventos recurrentes que alteran los patrones de flujo, ya que el plan fijo no se adapta a la ocupación real de las aproximaciones y puede generar tiempos de verde desaprovechados o demoras excesivas en ciertas corrientes \cite{papageorgiou2003review,stevanovic2010adaptive}.

Desde la perspectiva de la optimización, el diseño de planes de tiempos fijos puede formularse como un problema de programación entera mixta (Mixed-Integer Linear Programming, MILP) en el que las variables de decisión —duración del ciclo, tiempos de verde por fase (splits) y, en el caso de redes coordinadas, los desplazamientos temporales entre intersecciones (offsets)— son esencialmente discretas o deben satisfacer relaciones de discretización \cite{gartner1975multiband,wong1998optimizing}. En este marco, la función objetivo típicamente busca minimizar el retraso total de la red o maximizar la banda de progresión de pelotones a lo largo de un corredor, sujeta a restricciones de consistencia de ciclo, tiempos mínimos de verde, capacidad y sincronización \cite{gartner1975multiband,hillier1967mixed}. Los métodos analíticos clásicos, como las fórmulas de Webster \cite{webster1958traffic}, constituyen soluciones cerradas para versiones simplificadas de este problema: asumen una intersección aislada, demanda constante y condiciones de estado estacionario, permitiendo obtener expresiones analíticas para el ciclo óptimo y los splits que minimizan el retraso medio. Sin embargo, estas soluciones ignoran las interdependencias entre intersecciones, la naturaleza estocástica de los arribos y la presencia de restricciones operativas complejas \cite{wong1998optimizing,papageorgiou2003review}.

Cuando se considera la optimización de redes semafóricas con múltiples intersecciones coordinadas, el problema adquiere dimensiones combinatorias elevadas: el número de configuraciones posibles crece exponencialmente con el número de nodos, fases y rangos de valores admisibles para ciclos y offsets, lo que convierte al problema en NP--difícil \cite{gartner1975multiband,wong1998optimizing}. En estos casos, los métodos exactos de programación entera mixta pueden resultar computacionalmente intratables para redes de tamaño medio o grande, particularmente cuando se requiere resolver el problema en tiempo razonable para permitir ajustes frecuentes de los planes en respuesta a cambios de demanda. Por ello, la literatura ha recurrido a técnicas heurísticas y metaheurísticas —como algoritmos genéticos, simulated annealing y búsqueda local— para encontrar soluciones de buena calidad sin garantizar la optimalidad global \cite{park1999genetic,ceylan2004traffic}. En particular, los algoritmos genéticos han mostrado eficacia para optimizar planes de tiempos fijos en condiciones de demanda sobresaturada, donde los modelos analíticos tradicionales pierden validez y se requiere exploración robusta del espacio de soluciones \cite{park1999genetic}.

Diversos estudios en ingeniería de tráfico han mostrado que, en condiciones de demanda altamente variable, los esquemas de control fijo tienden a producir mayores tiempos de retraso medio, un mayor número de paradas y colas más extensas que los sistemas actuados o adaptativos \cite{papageorgiou2003review,stevanovic2010adaptive}. Este comportamiento motiva la búsqueda de estrategias de control más flexibles, como las basadas en lógica difusa o algoritmos de optimización metaheurística, que constituyen el foco del trabajo.


% -----------------------------------------------------------------------------


\subsection{Semáforos Accionados por el Tránsito}
\label{subsec:semaforos_accionados_transito}

Los semáforos accionados por el tránsito utilizan información en tiempo real captada por detectores (por ejemplo, lazos inductivos, sensores de radar o cámaras de video) ubicados en las aproximaciones para modificar la duración de las fases dentro de límites mínimos y máximos definidos \cite{mopc2019manual}. El Manual de Carreteras del Paraguay distingue entre control parcialmente actuado (al menos una aproximación bajo control por detectores) y totalmente actuado (todas las aproximaciones controladas por demanda), estableciendo criterios de diseño para la ubicación de detectores, la distancia respecto de la línea de detención y los períodos iniciales mínimos y extensiones de verde en función de la velocidad que comprende el 85\,\% del tránsito en el acceso \cite{mopc2019manual}.

En estos sistemas, el controlador decide en cada ciclo si extiende o termina el verde de una fase atendiendo a la detección de vehículos, lo que permite reducir tiempos de verde desaprovechados en movimientos con baja demanda y asignar capacidad adicional a las aproximaciones más cargadas \cite{roess2019traffic,papageorgiou2003review}. La literatura sobre control semafórico actuado y adaptativo destaca que este tipo de sistemas mejora el desempeño frente a planes fijos, especialmente en condiciones de demanda fluctuante, al disminuir el retraso medio por vehículo y el número de paradas, y al reducir la probabilidad de colas críticas en aproximaciones dominantes \cite{papageorgiou2003review,stevanovic2010adaptive}. 

El uso de detectores también permite implementar estrategias avanzadas como extensiones de verde para “limpiar” colas residuales, prioridad al transporte público o prioridad a vehículos de emergencia, resultando particularmente relevante en corredores urbanos con alta variabilidad de flujos \cite{papageorgiou2003review,gartner2001arterial}. Estos conceptos constituyen la base sobre la cual se articula el sistema de control inteligente propuesto en este trabajo, donde un controlador difuso tipo Mamdani y algoritmos de optimización como Particle Swarm Optimization (PSO) se integran para ajustar dinámicamente parámetros de operación semafórica en respuesta a condiciones de tráfico cambiantes.

Entre las técnicas metaheurísticas aplicadas al control semafórico, Particle Swarm Optimization (PSO) ha demostrado eficacia en la exploración del espacio de parámetros de señalización, gracias a su capacidad para escapar de óptimos locales y su relativa simplicidad de implementación \cite{kennedy1995particle,li2016pso,goncalo2022tuning}. Cuando se combina con sistemas de inferencia difusa tipo Mamdani \cite{mamdani1975experiment,musriroh2020application,trabia1999two}, que mapean variables lingüísticas de tráfico (congestión ``baja'', ``media'', ``alta'') a decisiones de control, se obtiene un enfoque híbrido capaz de incorporar conocimiento experto y adaptarse dinámicamente a condiciones variables, superando las limitaciones de los planes fijos y aproximándose al comportamiento de sistemas totalmente adaptativos. Este paradigma de control híbrido difuso--metaheurístico aprovecha la transparencia e interpretabilidad de los sistemas difusos para modelar relaciones complejas entre variables de tráfico y decisiones de control, mientras que PSO actúa como mecanismo de ajuste fino de los parámetros del sistema (funciones de pertenencia, pesos de reglas, ganancias de control), optimizando el desempeño global sin requerir modelos matemáticos explícitos del comportamiento del tráfico \cite{dellorco2013harmony,papageorgiou2003review}.


% -----------------------------------------------------------------------------


\subsection{Técnicas Algorítmicas y Computacionales para Control Adaptativo}
\label{subsec:tecnicas_algoritmicas}

La creciente disponibilidad de sensores de tráfico en tiempo real, el incremento de la capacidad de procesamiento computacional y los avances en inteligencia artificial han posibilitado el desarrollo de estrategias de control semafórico que van más allá de los enfoques clásicos basados en planes de tiempo fijo o en lógicas actuadas simples \cite{papageorgiou2003review,stevanovic2010adaptive}. Estas técnicas algorítmicas y computacionales permiten abordar el problema de optimización de tiempos semafóricos en condiciones de demanda estocástica y no estacionaria, superando las limitaciones de los métodos analíticos tradicionales que asumen flujos uniformes y equilibrio de estado \cite{papageorgiou2003review,li2016realtime}.

Entre las técnicas más consolidadas se encuentran los sistemas de control adaptativo en tiempo real, como SCOOT (Split, Cycle and Offset Optimization Technique) y SCATS (Sydney Coordinated Adaptive Traffic System), que ajustan automáticamente los parámetros de señalización a partir de información de ocupación y flujo captada por detectores ubicados aguas arriba de las intersecciones \cite{hunt1982scoot,sims1980scats}. SCOOT, desarrollado en el Reino Unido, emplea modelos de tráfico microscópicos para predecir la llegada de pelotones y ajusta ciclo, splits y offsets de forma continua, minimizando una función de costo basada en demoras y paradas \cite{hunt1982scoot,papageorgiou2003review}. SCATS, implementado originalmente en Australia, utiliza un enfoque de biblioteca de planes (plan library) en el que un controlador central selecciona dinámicamente el plan más apropiado de un conjunto predefinido en función de las condiciones de tráfico observadas, y ajusta los tiempos de verde dentro de límites establecidos \cite{sims1980scats,gartner2001arterial}. Ambos sistemas han demostrado reducciones significativas en demoras y emisiones en redes urbanas de mediana y gran escala, aunque requieren infraestructura de detección extensa y calibración cuidadosa de sus parámetros operativos.

Paralelamente, el campo de la optimización metaheurística ha aportado algoritmos capaces de explorar eficientemente el espacio de soluciones de problemas combinatorios complejos, evitando quedar atrapados en óptimos locales. Los algoritmos genéticos (Genetic Algorithms, GA) emulan procesos de evolución biológica —selección, cruce y mutación— para hacer evolucionar poblaciones de soluciones candidatas (configuraciones de ciclo, splits y offsets) hacia regiones de alta calidad del espacio de búsqueda, y han sido aplicados con éxito a la optimización de redes semafóricas coordinadas y a la optimización multiobjetivo que balancea demoras, emisiones y equidad \cite{ceylan2004traffic,papageorgiou2003review}. Particle Swarm Optimization (PSO), inspirado en el comportamiento social de bandadas de aves o cardúmenes de peces, representa cada solución candidata como una ``partícula'' que se desplaza en el espacio de parámetros guiada por su propia experiencia (mejor posición histórica individual) y por la experiencia del conjunto (mejor posición global del enjambre), logrando convergencia rápida y robusta en problemas de optimización de tiempos semafóricos \cite{kennedy1995particle,li2016pso,goncalo2022tuning}. Otros algoritmos metaheurísticos aplicados al dominio incluyen \emph{simulated annealing}, búsqueda tabú y optimización por colonias de hormigas, cada uno con fortalezas específicas en términos de diversidad de exploración, velocidad de convergencia y manejo de múltiples objetivos \cite{papageorgiou2003review}.

En el ámbito de la inteligencia artificial, los sistemas basados en lógica difusa (fuzzy logic) permiten incorporar conocimiento experto y razonamiento aproximado en el control de semáforos, modelando conceptos lingüísticos como ``congestión alta'', ``demanda moderada'' o ``cola larga'' mediante funciones de pertenencia y reglas de inferencia del tipo ``si–entonces'' \cite{mamdani1975experiment,musriroh2020application}. Los controladores difusos tipo Mamdani mapean variables de entrada continuas (por ejemplo, volumen, velocidad, densidad) a decisiones de control (extensión de verde, cambio de fase) de forma intuitiva y transparente, facilitando la interpretación y el ajuste de los parámetros por parte de operadores humanos \cite{ross2010fuzzy,erdinc2023application}. Las redes neuronales artificiales (Artificial Neural Networks, ANN) han sido exploradas como aproximadores universales de funciones complejas en el contexto de predicción de tráfico y control adaptativo, aunque su naturaleza de ``caja negra'' dificulta la interpretabilidad de las decisiones \cite{papageorgiou2003review}. Más recientemente, las técnicas de aprendizaje por refuerzo (Reinforcement Learning, RL) —en particular Q-learning y Deep Q-Networks (DQN)— han ganado prominencia al permitir que un agente de control aprenda políticas óptimas de asignación de verde a partir de la interacción directa con el entorno de tráfico, sin requerir un modelo explícito del sistema y adaptándose automáticamente a patrones de demanda cambiantes \cite{abdulhai2003reinforcement,li2016realtime}. Estos enfoques de aprendizaje profundo han mostrado resultados prometedores en simulaciones, superando en algunos casos a SCOOT y SCATS en escenarios de alta variabilidad, aunque aún enfrentan desafíos de estabilidad, seguridad y aceptación regulatoria para su despliegue en sistemas reales \cite{li2016realtime}.

Finalmente, enfoques más radicales como los sistemas autoorganizados proponen que cada intersección opere de forma autónoma, coordinándose con sus vecinas mediante intercambio de información local y reglas de decisión descentralizadas, sin depender de un controlador central \cite{cools2013selflearning}. Esta perspectiva se inspira en fenómenos naturales de autoorganización y busca mayor robustez frente a fallos y escalabilidad en redes extensas, aunque plantea interrogantes sobre garantías de desempeño y estabilidad global del sistema.

En conjunto, estas técnicas algorítmicas y computacionales representan una evolución desde los paradigmas tradicionales de control fijo y actuado hacia estrategias verdaderamente adaptativas, capaces de responder de forma dinámica y autónoma a la variabilidad inherente del tráfico urbano moderno. El presente trabajo se inscribe en esta línea al combinar un sistema de inferencia difusa tipo Mamdani —que captura conocimiento experto sobre congestión— con el algoritmo metaheurístico PSO —que optimiza los parámetros de operación semafórica—, conformando un enfoque híbrido que aprovecha las fortalezas de ambas técnicas y que será descrito en detalle en secciones posteriores.


% -----------------------------------------------------------------------------


\subsection{Parámetros Clásicos de Diseño de Tiempos Semafóricos}
\label{subsec:parametros_diseno_tiempos}

En el diseño clásico de control semafórico, la configuración de una intersección se describe mediante un conjunto reducido de parámetros numéricos que determinan su desempeño operativo: la longitud de ciclo, la repartición de verdes por fase (splits), los tiempos de amarillo y todo–rojo, y los tiempos perdidos en cada etapa del ciclo \cite{garber2019traffic,roess2019traffic}. La longitud de ciclo $C$ representa la duración total de una repetición completa de las fases del semáforo, mientras que los splits determinan qué fracción de ese ciclo se asigna al verde efectivo de cada movimiento o grupo de movimientos. Los tiempos de amarillo y todo–rojo se definen para advertir el fin del derecho de paso y garantizar el despeje seguro de la intersección antes de otorgar el verde a flujos conflictivos, y los tiempos perdidos agrupan intervalos en los que, por razones de seguridad o arranque, la capacidad efectiva de la intersección es menor a la teórica \cite{garber2019traffic}.

Estos parámetros influyen directamente en la capacidad y el retraso en cada aproximación. Para una misma demanda, ciclos más largos tienden a aumentar la capacidad por fase pero también pueden incrementar el tiempo medio de espera si los usuarios deben esperar largos intervalos de rojo, mientras que ciclos muy cortos reducen el retraso pero pueden volverse ineficientes si el tiempo perdido ocupa una proporción significativa del ciclo \cite{roess2019traffic,papageorgiou2003review}. La capacidad se relaciona con la proporción de verde asignada a cada movimiento y con la saturación de los carriles, en tanto que el retraso medio por vehículo, el número de paradas y la longitud de colas son métricas fundamentales para evaluar la calidad de servicio de un plan de tiempos \cite{garber2019traffic,fhwa2008signal}.

Métodos de diseño tradicional, como la formulación de Webster, proporcionan expresiones para estimar un ciclo ``óptimo'' fijo a partir de los volúmenes de diseño y de los tiempos perdidos, buscando un compromiso entre capacidad y retraso medio \cite{webster1958traffic,fhwa2008signal}. En la práctica, estos valores se calculan para condiciones de flujo representativas y se implementan como planes de tiempo fijo, eventualmente diferenciados por franja horaria. En contraste, el enfoque desarrollado en este trabajo utiliza un sistema de inferencia difusa tipo Mamdani y un algoritmo de optimización PSO para ajustar dinámicamente la longitud de ciclo efectiva, los splits y otros parámetros de operación en función de mediciones de tráfico en tiempo real, con el objetivo de reducir retrasos y mejorar el desempeño frente a condiciones de demanda variables \cite{papageorgiou2003review,stevanovic2010adaptive}.


% -----------------------------------------------------------------------------


\subsection{Modelado del Control Semafórico como Problema de Optimización}
\label{subsec:modelado_optimizacion}

El diseño y operación de sistemas de control semafórico puede formularse como un problema de optimización combinatoria en el que se busca determinar los valores de un conjunto de variables de decisión que minimicen (o maximicen) una función objetivo, respetando un conjunto de restricciones operativas y de seguridad \cite{wong1998optimizing,hillier1967mixed}. Esta perspectiva permite trascender los enfoques tradicionales basados en fórmulas empíricas y abordar de manera sistemática la complejidad inherente a redes urbanas con demanda variable y múltiples intersecciones interdependientes.

En su formulación más general, el problema de optimización de control semafórico define como variables de decisión los parámetros temporales que gobiernan la operación del sistema: la duración del ciclo $C$, los tiempos de verde efectivo asignados a cada fase o movimiento (splits $g_i$), los desplazamientos temporales entre intersecciones consecutivas (offsets $\phi_j$), y en algunos casos la secuencia misma de fases \cite{wong1998optimizing,garber2019traffic}. Estas variables deben elegirse dentro de rangos factibles definidos por la geometría, la capacidad de las aproximaciones y los requerimientos normativos de tiempos mínimos y máximos.

La función objetivo del problema representa la métrica que se desea optimizar y puede adoptar diversas formas según el criterio de desempeño que se privilegie. Entre las formulaciones más comunes se encuentran la minimización del retraso total experimentado por todos los vehículos en la red, la minimización del número total de paradas, la maximización del throughput (número de vehículos que atraviesan el sistema por unidad de tiempo), o la minimización de emisiones contaminantes y consumo de combustible derivados de aceleraciones y frenados \cite{wong1998optimizing,stevanovic2010adaptive,papageorgiou2003review}. En aplicaciones recientes se han propuesto funciones objetivo multicriterio que balancean simultáneamente eficiencia operativa, equidad entre corrientes de tráfico, seguridad y sostenibilidad ambiental, empleando técnicas de optimización multiobjetivo como frentes de Pareto o funciones de utilidad ponderadas \cite{ceylan2004traffic,stevanovic2010adaptive}.

El problema se completa con un conjunto de restricciones que garantizan la factibilidad y seguridad de las soluciones. Estas restricciones incluyen tiempos mínimos de verde para permitir que los peatones crucen la intersección de forma segura, tiempos máximos de verde para evitar demoras excesivas en movimientos conflictivos, tiempos de amarillo y todo–rojo calculados en función de la velocidad de aproximación y las distancias de despeje, y restricciones de capacidad que limitan los flujos admisibles en cada carril o aproximación \cite{garber2019traffic,fhwa2008signal}. En el caso de redes coordinadas, se añaden restricciones de consistencia de ciclo (todas las intersecciones deben operar con el mismo ciclo común o con múltiplos enteros del ciclo base) y restricciones de sincronización que determinan los valores admisibles de los offsets para lograr progresión de pelotones \cite{erdogan2013coordination,gartner2001arterial}.

Los métodos analíticos tradicionales, como las fórmulas de Webster \cite{webster1958traffic} o el método del Highway Capacity Manual (HCM) \cite{roess2019traffic}, constituyen soluciones cerradas para versiones simplificadas del problema de optimización, en las que se asumen condiciones de estado estacionario, demanda uniforme y geometrías regulares. Estas formulaciones proporcionan estimaciones rápidas del ciclo óptimo y de los splits correspondientes, y resultan útiles para el diseño preliminar de planes de tiempo fijo. Sin embargo, en la práctica, las condiciones reales de tráfico urbano se caracterizan por una alta variabilidad temporal (diferencias entre horas pico y valle, fluctuaciones aleatorias de la demanda), heterogeneidad espacial (diferencias de volumen entre aproximaciones), y la presencia de eventos no recurrentes (accidentes, obras, eventos especiales) que invalidan las hipótesis de estacionariedad de los modelos analíticos \cite{papageorgiou2003review,stevanovic2010adaptive}.

Esta variabilidad convierte al problema de optimización semafórica en un desafío computacional complejo, especialmente cuando se considera la operación coordinada de múltiples intersecciones a lo largo de corredores o redes completas. En estos casos, el espacio de búsqueda de soluciones crece exponencialmente con el número de intersecciones y de fases, y el problema adquiere la estructura de un problema de optimización combinatoria NP--difícil, para el cual no se conocen algoritmos exactos con tiempo de ejecución polinomial \cite{wong1998optimizing,ceylan2004traffic}. Ante esta complejidad, las técnicas algorítmicas avanzadas —como algoritmos genéticos, Particle Swarm Optimization (PSO), simulated annealing y métodos de búsqueda local— se han consolidado como herramientas eficaces para explorar el espacio de soluciones, escapar de óptimos locales y encontrar configuraciones de alta calidad en tiempos de cómputo razonables \cite{ceylan2004traffic,papageorgiou2003review}.

En síntesis, la formulación del control semafórico como problema de optimización combinatoria proporciona un marco conceptual riguroso que permite integrar distintos criterios de desempeño, considerar múltiples restricciones operativas y de seguridad, y justificar el empleo de técnicas computacionales avanzadas —como los sistemas de inferencia difusa y los algoritmos metaheurísticos— que serán presentados en secciones posteriores y que constituyen la base del sistema de control inteligente desarrollado en este trabajo.


% -----------------------------------------------------------------------------


\subsection{Coordinación de Semáforos y Progresión de Pelotones}
\label{subsec:coordinacion_semaforos}

En redes urbanas, los semáforos rara vez operan de manera completamente aislada; por el contrario, se busca coordinar varios equipos a lo largo de un corredor para facilitar la progresión de pelotones de vehículos y reducir el número de paradas sucesivas \cite{garber2019traffic,roess2019traffic}. La coordinación se logra ajustando parámetros como la longitud de ciclo común entre intersecciones, los offsets (desplazamientos temporales entre inicios de verde) y, en algunos casos, la secuencia de fases, de modo que un grupo de vehículos que parte con verde en una intersección encuentre verdes sucesivos en las siguientes, configurando la llamada ``onda verde'' \cite{fhwa2008signal,erdogan2013coordination}. Estos esquemas buscan compatibilizar la operación de múltiples nodos de control, sacrificando a veces el óptimo local de una intersección en beneficio del desempeño global del corredor.

La coordinación introduce dependencias fuertes entre los parámetros de distintas intersecciones: cambios en la longitud de ciclo o en los splits de una intersección pueden degradar la progresión en tramos adyacentes, generando nuevas demoras o incrementando el número de paradas en el corredor \cite{erdogan2013coordination,roess2019traffic}. Por ello, el diseño de planes coordinados suele apoyarse en herramientas de simulación y optimización que consideran simultáneamente varias intersecciones, y en prácticas de monitoreo operativo que permiten ajustar offsets y ciclos en respuesta a cambios de demanda. En este contexto, la idea de un sistema de control y monitoreo distribuido, apoyado en tecnologías como BlockDAG e IPFS para registrar configuraciones y métricas de desempeño, resulta especialmente pertinente: aunque el presente trabajo se centre en la optimización a nivel de una intersección, la misma arquitectura puede escalar a corredores coordinados donde las decisiones de tiempo de verde, ciclo y offset deben gestionarse de forma conjunta y trazable.


% -----------------------------------------------------------------------------


\subsection{Consideraciones Operativas y Métricas de Desempeño}
\label{subsec:metricas_desempeno}

La evaluación de un plan de tiempos semafóricos se basa en un conjunto de métricas operativas que permiten cuantificar su impacto sobre los usuarios de la vía. Entre las más utilizadas se encuentran el retraso medio por vehículo, el número de paradas, la longitud de colas y el nivel de servicio (Level of Service, LOS); en estudios recientes se añaden además indicadores de consumo de combustible y emisiones contaminantes \cite{garber2019traffic,roess2019traffic,papageorgiou2003review}. El retraso medio mide la diferencia entre el tiempo de viaje real y el tiempo teórico sin interferencias, mientras que el número de paradas y la longitud de colas caracterizan la discontinuidad del flujo y la probabilidad de bloqueo de accesos o de intersecciones adyacentes \cite{garber2019traffic,fhwa2008signal}.

El nivel de servicio resume de forma cualitativa la calidad de operación en categorías que van desde ``A'' (pocas demoras, operación confortable) hasta ``F'' (congestión severa), definidas a partir de rangos de retraso medio por vehículo y otras variables \cite{roess2019traffic,fhwa2008signal}. Estas métricas permiten comparar diferentes esquemas de control (planes fijos, control actuado, control adaptativo) bajo condiciones de demanda similares y constituyen la base para justificar cambios en la programación de semáforos desde una perspectiva de ingeniería.

En el contexto de este trabajo, el desempeño de los planes semafóricos se evalúa a partir de indicadores derivados de la salida del sistema difuso de congestión (valor numérico de congestión y su categoría lingüística), junto con variables macroscópicas como volumen por minuto, velocidad media y densidad vehicular. El controlador difuso Mamdani proporciona una medida agregada de congestión a partir de estas variables, y el algoritmo PSO ajusta los tiempos de verde para minimizar directamente dicho índice de congestión y mejorar las condiciones de flujo respecto a una asignación base de tiempos calculada mediante fórmulas clásicas de ingeniería de tráfico. La automatización de la recolección de estas métricas y su análisis en tiempo real plantea desafíos de almacenamiento, trazabilidad e integridad de datos. En sistemas de control inteligente distribuido, donde múltiples nodos (intersecciones) operan de forma coordinada, resulta fundamental contar con mecanismos que aseguren la auditabilidad de las decisiones de control y la inmutabilidad de los registros históricos de configuración y desempeño \cite{singh2020blockchain,yuan2018blockchain}. Este requisito motiva el uso de arquitecturas descentralizadas como BlockDAG e IPFS, que serán analizadas en secciones posteriores.


% -----------------------------------------------------------------------------


\subsection{Integración de Técnicas de Optimización y Control Difuso}
\label{subsec:integracion_optimizacion_difuso}

La convergencia entre técnicas de optimización metaheurística y sistemas de inferencia difusa ha dado lugar a enfoques híbridos que combinan la capacidad de razonamiento aproximado y manejo de incertidumbre de la lógica difusa con la potencia de búsqueda y ajuste paramétrico de algoritmos como Particle Swarm Optimization (PSO), algoritmos genéticos y otras metaheurísticas \cite{papageorgiou2003review,dellorco2013harmony}. En el contexto del control semafórico, estos sistemas híbridos fuzzy--metaheurísticos permiten abordar de manera integrada dos desafíos complementarios: por un lado, la captura y formalización del conocimiento experto sobre las relaciones entre variables de tráfico y decisiones de control mediante reglas lingüísticas y funciones de pertenencia difusas; por otro, la optimización automática de los parámetros del sistema difuso (formas y umbrales de las funciones de pertenencia, pesos de las reglas, ganancias de salida) de forma que se maximice una función objetivo operativa, como la minimización de retrasos o la maximización del throughput de la red \cite{azimirad2010novel,yadav2021fuzzy}.

En un sistema de control difuso tipo Mamdani aplicado a semáforos, las funciones de pertenencia de las variables de entrada (por ejemplo, volumen vehicular, velocidad media, densidad, longitud de cola) y de salida (extensión de verde, ajuste de ciclo, cambio de fase) se definen inicialmente con base en conocimiento experto o en datos históricos, pero su sintonización fina puede requerir ajustes iterativos costosos y poco sistemáticos. PSO ofrece una alternativa eficiente y automática: cada partícula del enjambre representa una configuración específica de parámetros del sistema difuso, y la posición de la partícula se actualiza iterativamente en función de su desempeño histórico y del desempeño del mejor individuo global, explorando el espacio de configuraciones hasta encontrar un conjunto de parámetros que optimice la métrica de desempeño deseada \cite{kennedy1995particle,li2016pso,goncalo2022tuning}. Este enfoque de optimización de parámetros difusos mediante PSO ha demostrado reducir significativamente el tiempo de diseño del controlador, mejorar la robustez frente a variaciones de demanda y permitir la adaptación online de parámetros en respuesta a condiciones cambiantes \cite{dellorco2013harmony,azimirad2010novel}.

Extensiones más complejas incluyen sistemas neuro--fuzzy, en los que se combinan redes neuronales artificiales con lógica difusa para permitir aprendizaje automático de las reglas de inferencia y funciones de pertenencia a partir de datos, y que pueden ser entrenados o ajustados mediante metaheurísticas como PSO, algoritmos genéticos o búsqueda cuco (Cuckoo Search) \cite{nawi2013fuzzy}. Estos enfoques híbridos neuro--fuzzy--metaheurísticos aprovechan la capacidad de generalización y aprendizaje de las redes neuronales, la interpretabilidad de los sistemas difusos y la eficiencia de búsqueda de las metaheurísticas, constituyendo una vía prometedora para el diseño de controladores adaptativos en dominios complejos como el control de tráfico urbano \cite{papageorgiou2003review,yadav2021fuzzy}.

El sistema propuesto en este trabajo se inscribe en esta línea de sistemas híbridos fuzzy--PSO: emplea un controlador difuso tipo Mamdani para evaluar el nivel de congestión a partir de variables macroscópicas de tráfico (volumen, velocidad, densidad) y utiliza PSO para optimizar los tiempos de verde de las fases semafóricas de forma que se minimice el índice de congestión estimado por el sistema difuso. Este enfoque permite incorporar conocimiento experto sobre la relación entre variables de tráfico y estados de congestión en las reglas difusas, mientras que PSO asegura que la asignación de tiempos de verde sea óptima respecto de la configuración de demanda observada. La arquitectura resultante es adaptativa, interpretable y computacionalmente eficiente, características esenciales para su despliegue en entornos operativos reales.

Más allá de las ventajas operativas del enfoque híbrido, la implementación de sistemas de control inteligente en infraestructuras críticas como redes semafóricas plantea requisitos adicionales de trazabilidad, auditabilidad e integridad de datos. En un sistema distribuido donde múltiples intersecciones operan de forma coordinada o autónoma, resulta fundamental registrar de manera inmutable y verificable las decisiones de control tomadas (tiempos de verde asignados, parámetros del controlador difuso, resultados de la optimización PSO), las condiciones de tráfico observadas (volúmenes, velocidades, densidades capturadas por sensores) y las métricas de desempeño resultantes (retrasos, emisiones, niveles de servicio). Estos registros permiten auditar el comportamiento del sistema, identificar configuraciones problemáticas, validar la conformidad con normativas y políticas de operación, y facilitar la depuración y mejora continua del controlador. Las arquitecturas descentralizadas basadas en BlockDAG (Directed Acyclic Graph de bloques) e IPFS (InterPlanetary File System) ofrecen soluciones escalables y confiables para el registro y almacenamiento distribuido de estos datos, temas que serán analizados en profundidad en secciones posteriores del presente trabajo. En síntesis, la integración de técnicas de optimización metaheurística y control difuso no solo mejora el desempeño operativo del sistema semafórico, sino que abre la puerta a arquitecturas distribuidas y auditables que responden a las demandas de transparencia y confiabilidad de las ciudades inteligentes modernas.
