% =============================================================================
% SECCIÓN 2.6: PARTICLE SWARM OPTIMIZATION (PSO)
% =============================================================================

\subsection{Particle Swarm Optimization (PSO)}
\label{sec:pso}

Particle Swarm Optimization (PSO) es un metaheurístico de optimización global inspirado en el comportamiento colectivo de bandadas de aves y bancos de peces, en el que un conjunto de partículas intercambia información sobre buenas soluciones para desplazarse en el espacio de búsqueda \cite{kennedy1995particle}. En ingeniería de tráfico, PSO se ha aplicado con éxito al ajuste de tiempos semafóricos y otros parámetros de control, mostrando reducciones relevantes en demoras y tiempos de espera en simulaciones de intersecciones \cite{li2016pso,goncalo2022tuning}.

\subsubsection{Algoritmo PSO: Inspiración y Dinámica Básica}

PSO fue introducido por Kennedy y Eberhart en 1995 como un método para optimizar funciones no lineales mediante un conjunto de partículas que se mueven en un espacio de búsqueda multidimensional \cite{kennedy1995particle}. Cada partícula representa una solución candidata, caracterizada por una posición $x_i$ y una velocidad $v_i$, que se actualizan en función de la experiencia propia y de la del grupo.

La inspiración biológica proviene de modelos de comportamiento social en bandadas, en los que cada individuo ajusta su trayectoria considerando tanto su mejor experiencia previa como la de sus vecinos o la del enjambre completo. En PSO, esta idea se formaliza mediante dos componentes: el componente cognitivo, que representa la tendencia de la partícula a regresar a su mejor posición individual conocida (pbest), y el componente social, que representa la tendencia a acercarse a la mejor posición encontrada por el grupo (gbest) o por su vecindario en variantes locales \cite{eberhart1995new}. En su forma canónica, la actualización combina una componente de inercia, una atracción hacia la mejor posición personal y otra hacia la mejor posición global, ponderadas por parámetros $w$, $c_1$ y $c_2$, y términos aleatorios $r_1$ y $r_2$ \cite{clerc2002particle}.

La Figura~\ref{fig:pso_enjambre} ilustra esquemáticamente el concepto del enjambre PSO en un espacio de búsqueda bidimensional, mostrando cómo las partículas se mueven influenciadas por su mejor experiencia personal y la mejor experiencia global del grupo.

\begin{figure}[htbp]
    \centering
    \shorthandoff{>}
    \begin{tikzpicture}[scale=1.8]
        % Ejes
        \draw[->,line width=1.3pt] (-0.2,0) -- (6.2,0) node[below,font=\Large] {$x_1$};
        \draw[->,line width=1.3pt] (0,-0.2) -- (0,4.2) node[left,font=\Large] {$x_2$};
        
        % Partículas
        \fill[red] (1.0,3.0) circle (2.5pt) node[above left,font=\large] {$\mathbf{x}_1$};
        \fill[red] (2.0,2.5) circle (2.5pt) node[above,font=\large] {$\mathbf{x}_2$};
        \fill[red] (4.5,2.2) circle (2.5pt) node[above right,font=\large] {$\mathbf{x}_3$};
        
        % Mejor personal de una partícula
        \fill[orange] (1.8,3.4) circle (2.5pt) node[above,font=\large] {$p_1$};
        \draw[->,orange,dashed,line width=1.8pt] (1.0,3.0) -- (1.8,3.4);
        
        % Mejor global
        \fill[green!70!black] (3.0,1.6) circle (3pt) node[below right,font=\large] {$g$};
        \draw[->,green!70!black,dashed,line width=1.8pt] (2.0,2.5) -- (3.0,1.6);
        \draw[->,green!70!black,dashed,line width=1.8pt] (4.5,2.2) -- (3.0,1.6);
        
        % Leyenda
        \node[draw,fill=white,anchor=west] at (6.5,2.0) {%
          \begin{tabular}{@{}l@{}}
            \textcolor{red}{$\bullet$} Partículas (soluciones)\\
            \textcolor{orange}{$\bullet$} Mejor personal $p_i$\\
            \textcolor{green!70!black}{$\bullet$} Mejor global $g$
          \end{tabular}
        };
    \end{tikzpicture}
    \shorthandon{>}
    \caption{Esquema conceptual del enjambre PSO en un espacio de búsqueda bidimensional, mostrando partículas, mejor posición personal ($p_i$) y mejor posición global ($g$).}
    \label{fig:pso_enjambre}
\end{figure}

\subsubsection{Parámetros Principales: $w$, $c_1$, $c_2$ y Tamaño de Población}

El comportamiento de PSO está fuertemente gobernado por cuatro parámetros clave: el peso de inercia $w$, los coeficientes de aceleración $c_1, c_2$ y el tamaño de la población (número de partículas). El peso de inercia $w$ controla cuánto de la velocidad anterior se preserva en cada actualización, modulando el equilibrio entre exploración y explotación: valores altos de $w$ favorecen recorridos más largos y exploración global, mientras que valores más bajos refuerzan la explotación local y aceleran la convergencia \cite{shi1998parameter,clerc2002particle}.

Los parámetros $c_1$ (cognitivo) y $c_2$ (social) ponderan la atracción hacia la mejor posición personal y la mejor posición global, respectivamente. Un $c_1$ alto refuerza el comportamiento individual, promoviendo una búsqueda más dispersa al permitir que cada partícula explore alrededor de su propia experiencia; un $c_2$ alto incrementa la influencia del mejor miembro del grupo, favoreciendo una convergencia más rápida hacia la región más prometedora, aunque con riesgo de convergencia prematura si se elige de forma excesiva \cite{clerc2002particle}.

El tamaño del enjambre determina el grado de muestreo del espacio de búsqueda y el costo computacional por iteración. Poblaciones pequeñas reducen el costo computacional, pero pueden cubrir de forma insuficiente el espacio y tienen mayor probabilidad de caer en óptimos locales; poblaciones grandes mejoran la diversidad y la probabilidad de encontrar zonas promisorias, a costa de mayor tiempo de cómputo por iteración. Estudios empíricos sugieren que tamaños de enjambre en el rango de 20--60 partículas son adecuados para muchos problemas de dimensión moderada \cite{poli2007particle}.

\subsubsection{Convergencia y Optimización Global}

Aunque PSO se diseñó originalmente como un algoritmo heurístico, se ha desarrollado una literatura extensa que estudia su estabilidad y convergencia interpretándolo como un sistema dinámico discreto \cite{clerc2002particle,vandenbergh2010convergence,chen2024convergence}. Estos trabajos muestran que combinaciones adecuadas de $w$, $c_1$ y $c_2$ permiten trayectorias acotadas y convergencia hacia regiones cercanas al óptimo, mientras que configuraciones extremas pueden provocar oscilaciones o divergencia. En esta tesis se emplean configuraciones de parámetros recomendadas en la literatura, sin profundizar en el análisis formal de convergencia, ya que el foco se centra en la aplicación de PSO al problema de optimización de tiempos semafóricos.

\subsubsection{Aplicaciones en Optimización de Tráfico y Tiempos Semafóricos}

En el ámbito de sistemas de transporte inteligente, PSO se ha utilizado para optimizar planes de señalización semafórica, coordinación de fases y otros parámetros de control de tráfico en redes urbanas. La idea central es codificar los tiempos de verde/ámbar/rojo (y eventualmente offsets entre intersecciones) como variables de decisión en las partículas, y minimizar funciones objetivo relacionadas con el desempeño del tráfico, tales como el retraso medio por vehículo, el número de paradas o métricas combinadas que incluyen emisiones \cite{li2016pso,goncalo2022tuning}.

Para el caso de optimización de tiempos de semáforos en una intersección o red pequeña, una partícula suele codificar un vector del tipo $x_i = (g_1, g_2, \dots, g_M, \text{offset}_1, \dots)$, donde $g_k$ es el tiempo de verde de la fase $k$. Se imponen restricciones como límites mínimos y máximos por fase ($g_k^{\min} \le g_k \le g_k^{\max}$) y suma de tiempos por ciclo $\sum_k g_k = C$ o dentro de un rango $[C_{\min}, C_{\max}]$, manejadas mediante proyección al espacio factible o penalización en la función objetivo \cite{li2016pso}. La evaluación de cada partícula se realiza típicamente acoplando PSO con simuladores microscópicos como SUMO o VISSIM, que proporcionan métricas de desempeño (retraso, longitud de cola, número de paradas) para el plan semafórico propuesto \cite{goncalo2022tuning}.

Trabajos como el de Li et al.\ proponen enfoques de PSO para la combinación de fases y tiempos de verde en intersecciones, minimizando el número de paradas y el retraso total; en sus experimentos se reportan reducciones cercanas al 19\,\% en el número de paradas frente a un plan fijo base \cite{li2016pso}. Por su parte, Gonçalo y colaboradores integran PSO con SUMO para ajustar los tiempos de luz verde y obtienen reducciones del orden del 26\,\% en el tiempo medio de espera en comparación con configuraciones iniciales \cite{goncalo2022tuning}. Estos resultados posicionan a PSO como una herramienta flexible y robusta para la optimización de control de tráfico en escenarios donde la función objetivo se evalúa mediante simulación y carece de gradiente analítico.

